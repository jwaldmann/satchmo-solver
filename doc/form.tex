\section{Representation of Data in the Solver}

In the classical approach, the solver has a state that is held 
in global (or instance-specific) arrays, and which is modified.
The state consists mainly of (TODO: verify this)
clever pointer and index structures that refer to a set of clauses
(e.g., the ``two watched literals'' scheme to quickyl detect unit clauses
when they appear as a result of propagations).

In the pure approach, the ``current'' set of clauses is represented
by a persistent map (a collection of such). We still need ``clever pointing'' 
because we absolutely should avoid walking all elements of the map.

The basic model is to view the clause set as a bipartite graph
$(G_1,G_2,E)$ where edges are 2-coloured:
the vertices are partitioned into variables and clauses,
and there is an edge from $v$ to $c$ with colour $b$,
if $v$ occurs in $c$ with polarity $b$ 
(e.g., positive: colour ``+'', negative : colour ``-'')

This can be represented in Haskell as follows
\begin{verbatim}
newtype V = V Int deriving (Enum, Show, Eq, Ord)
newtype C = C Int deriving (Enum, Show, Eq, Ord)
data Form  =
    Form { fore :: ! (M.Map V ( M.Map C Bool ))
         , back :: ! (M.Map C ( M.Map V Bool ))
         }
\end{verbatim}

