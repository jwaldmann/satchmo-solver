\section{Representation of Data in the Solver}

In the classical approach, the solver has a state that is held 
in global (or instance-specific) arrays, and which is modified.
The state consists mainly of (TODO: verify this)
clever pointer and index structures that refer to a set of clauses
(e.g., the ``two watched literals'' scheme to quickyl detect unit clauses
when they appear as a result of propagations).

In the pure approach, the ``current'' set of clauses is represented
by a persistent map (a collection of such). We still need ``clever pointing'' 
because we absolutely should avoid walking all elements of the map.

The basic model is to view the clause set as a bipartite graph
$(G_1,G_2,E)$ where edges are 2-coloured:
the vertices are partitioned into variables and clauses,
and there is an edge from $v$ to $c$ with colour $b$,
if $v$ occurs in $c$ with polarity $b$ 
(e.g., positive: colour ``+'', negative : colour ``-'')

This can be represented in Haskell as follows
\begin{verbatim}
newtype V = V Int deriving (Enum, Show, Eq, Ord)
newtype C = C Int deriving (Enum, Show, Eq, Ord)
data Form  =
    Form { fore :: ! (M.Map V ( M.Map C Bool ))
         , back :: ! (M.Map C ( M.Map V Bool ))
         }
\end{verbatim}
This representation allows to efficiently
\begin{itemize}
\item given the variable, find all the clauses where it appears
\item given the clause, find all the variables that appear in it
\end{itemize}

This is used, e.g., in assigning a variable
\begin{verbatim}
assign :: (V, Bool) -> Form -> Form
assign (v, b) f =
  let cpos = M.filter (==b) $ fore f M.! v
      cneg = M.filter (/=b) $ fore f M.! v
      back' = foldr ( \ c m -> M.adjust (M.delete v) c m )
          (back f) (M.keys cneg)
      g = f { fore = M.delete v $ fore f 
            , back = back'
            }
  in  foldr drop_clause g (M.keys cpos)
\end{verbatim}

The representation does not allow to find unit clauses efficiently,
we can onyl write
\begin{verbatim}
units f = M.filter ( \ m -> 1 == M.size m ) $ back f
\end{verbatim}
which visits all clauses. This must clearly by improved,
and one way to do this, is to add another index.
\begin{verbatim}
data Form  =
    Form { ...
         , by_size :: M.Map Int (Set C)
         }
\end{verbatim}
with the invariant that (roughly)
\begin{verbatim}
(c,m)  `elem`  M.toList back f <==>  (c,M.size m) `elem` M.toList by_size f
\end{verbatim}
