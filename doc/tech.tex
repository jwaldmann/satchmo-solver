\section{Haskell Technicalities}

This section is for discussion of implementation detail
that is not interesting for SAT.

\subsection{EnumMap}

The \verb|containers| package provides the \verb|Data.IntMap| module
(big-endian patricia trees) and the type, and some methods, are:
\begin{verbatim}
module Data.IntMap where
data IntMap a
type Key = Int
lookup :: Key -> IntMap a -> Maybe a 
insert :: Key -> a -> IntMap a -> IntMap a 
unionWith :: (a -> a -> a) -> IntMap a -> IntMap a -> IntMap a 
\end{verbatim}
This directly represents the underlying idea: keys are indeed numbers,
and their binary representation is used for efficient implementation.

It is not so nice from a programmer's standpoint.
We want to represent a bipartite graph.
We want to distinguish the nodes (variables or clauses) by type:
\begin{verbatim}
newtype V = V Int deriving (Enum, Show, Eq, Ord)
newtype C = C Int deriving (Enum, Show, Eq, Ord)
\end{verbatim}
So we want to use \verb|V| and \verb|C| as keys for IntMaps,
but we cannot, as the key type is not polymorphic.
Well, we can, by introducing a wrapper
\begin{verbatim}
module Data.EnumMap where
import qualified Data.IntMap as IM
newtype Map k v = Map (IM.IntMap v)  deriving (Eq, Ord)
insert :: Enum a => a -> v -> Map t v -> Map k v
insert k v (Map m) = Map $ IM.insert (fromEnum k) v m
union :: Map t v -> Map t1 v -> Map k v
union (Map m1) (Map m2) = Map $ IM.union m1 m2
\end{verbatim}
Here, \verb|k| acts as a phantom type, it is only used 
to select the \verb|Enum| instance for wrapping and unwrapping the keys.
